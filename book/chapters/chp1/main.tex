% Write the full path to the location of the graphics relative to book.tex
\graphicspath{{chapters/chp1/graphics/}}
\title{A CUDA extension for FEniCSx: CUDOLFINx}
\titlerunning{A CUDA extension for FEniCSx: CUDOLFINx}

\author{Benjamin~A.~Pachev, James~D.~Trotter, and Igor~A.~Baratta}
\authorrunning{Pachev et al.}
\institute{Benjamin.~A.~Pachev \email{benjmainpachev@utexas.edu} \at The University of Texas at Austin \\James.~D.~Trotter \email{james@simula.no} \at Simula Research Laboratory
\\Igor~A.~Baratta \email{ia397@cam.ac.uk} \at Department of Engineering, Univeristy of Cambridge
}


\maketitle

\abstract{Here we introduce CUDOLFINx - a Python package which extends FEniCSx with GPU accelerated assembly capabilities. The extension enables FEniCSx codes to be accelerated on the GPU with minimal changes, and provides an easy path for researchers to experiment with GPU-accelerated PDE solvers. By constrast with previous efforts to enhance FEniCSx with GPU capabilities, CUDOLFINX is designed as a standalone package and does not require major changes to the core components of FEniCSx. Consequently, it has the potential to become a usable part of the FEniCSx ecosystem and a long-term solution to the problem of providing GPU acceleration capabilities in FEniCSx.
We further present performance benchmarks for a representative range of GPU-accelerated FEniCSx applications on an NVIDIA H100 GPU. Our results indicate that GPU-accelerated assembly routines within CUDOLFINX can be up to 40 times faster than traditional FEniCSx assembly with MPI-parallelization on a many-core CPU. In general, acceleration of assembly routines with CUDOLFINx results in the performance equivalent of an MPI cluster with several hundred to thousands of cores.
}

\section*{Introduction}

Graphics processing units (GPUs) provide an alternative means of parallelizing computations to traditional parallelism across clusters of multi-core CPUs. For many applications, GPUs are more energy-efficient and have revolutionized fields such as machine learning \citep{navarro2014survey}. However, the application of GPU acceleration to PDE solvers has yet to become the norm. Many PDE libraries such as FEniCSx \citep{baratta2023dolfinx}, NGSolve \citep{schoberl2014c++}, FreeFem++\citep{hecht2012new}, nektar++ \citep{moxey2020nektar++} and FireDrake \citep{FiredrakeUserManual} lack support for GPU acceleration. A major reason for this is the difficulty of modifying existing codes to use GPU parallelism \citep{MILLS2021102831}. GPU programming requires a specialized compiler, memory space, and syntax, meaning that code must often be largely rewritten to take advantage of GPU acceleration. Despite these challenges, a large number of software packages used for solving PDEs have taken steps to provide GPU acceleration capabilities. These include PETSc \citep{MILLS2021102831}, MFEM \citep{anderson2021mfem}, libCEED \citep{abdelfattah2021gpu}, and deal.II \citep{arndt2021deal}. Nevertheless, even when GPU acceleration is available within a library, it often has limited support and is difficult to use.
%Furthermore, parallelization schemes suitable for clusters of multi-core CPUs frequently don't work for GPUs \citep{MACIOL20101093}. Maximizing GPU performance often requires a customized algorithm, which further increases the difficulty of integrating GPU acceleration into existing projects \citep{mittal2015survey}.

A major attraction of FEniCSx \citep{baratta2023dolfinx} as a tool for solving partial differential equations with the finite element method is its simple Python interface and automated generation of efficient C code. These features enable rapid prototyping and development of performant solvers for complicated PDEs. Our goal in developing CUDOLFINx is to enable users of FEniCSx to add GPU acceleration to their existing PDE solvers with minimal effort, similar to how FEniCSx automatically handles MPI parallelization. GPUs are attractive for PDEs due to their increased throughput, floating point operations per second (FLOPS), and memory bandwidth. Although many PDEs are generally memory-bound, as will be seen later, even memory-bound problems can benefit significantly from GPU acceleration. During the remainder of this chapter, we will give a brief overview of CUDOLFINx, present some example applications to the Poisson, Navier--Stokes, and shallow water problems, and finally discuss future development of CUDOLFINx.

\section*{Overview of CUDOLFINx}

The two most expensive steps in the finite element method are linear solves and the assembly of matrices or vectors. We would like to accelerate both of these steps with GPUs, in part to minimize expensive copies to and from GPU memory. Libraries such as PETSc \citep{MILLS2021102831},  Ginkgo \citep{ginkgo-toms-2022}, AMGx \citep{naumov2015amgx}, hypre \citep{li2020efficient, falgout2021porting}, SuperLU \citep{li2023newly}, and others \citep{lu2023tilesptrsv} provide efficient GPU-acclerated linear solvers. The goal of CUDOLFINx is to enable GPU-accelerated assembly so that the entirety of FEniCSx finite element workflows can be GPU accelerated. Assembly routines within FEniCSx rely on auto-generated kernels which produce element tensors that can be assembled into a global matrix or vector. CUDOLFINx modifies these kernels so they can be executed on the GPU, and provides GPU assembly loops that assemble the local contributions from each element into global tensors. This requires that information about the mesh, boundary conditions, and function spaces be copied to GPU memory. Internally, CUDOLFINx has GPU versions of many of the data structures within FEniCSx. However, these data structures and the copying of information to the GPU are mostly hidden from the user. Using CUDOLFINx within existing FEniCSx code requires only minor modifications, as illustrated by the following code snippet:

\begin{python}
import cudolfinx as cufem

# given UFL forms A and L representing a stiffness matrix and right-hand-side
cuda_A = cufem.form(A)
cuda_L = cufem.form(L)
asm = cufem.CUDAAssembler()
# returns a custom type CUDAMatrix
mat = asm.assemble_matrix(cuda_A)
# get PETSc matrix
petsc_mat = mat.mat()
# returns a custom type CUDAVector
vec = asm.assemble_vector(cuda_L)
#get PETSc vector
petsc_vec = vec.vector()
\end{python}

The GPU acceleration scheme within CUDOLFINx is an extension of the work of \cite{trotter2023targeting}. The generated tabulate tensor routines from FFCx are used as-is, with no GPU-targeted changes. Each element is processed by a single GPU thread, and atomic operations are used to prevent data races. This approach works well for low-order elements, but is not appropriate for high-order methods, as these necessitate the use of multiple GPU threads per element to obtain an efficient algorithm \citep{MACIOL20101093,dziekonski2013generation,banas2014numerical}. We will proceed to demonstrate the peformance of CUDOLFINx for several representative use cases in which GPU acceleration can significantly enhance FEniCS code.

\section*{Performance Evaluation}

We begin with two examples of the performance of the GPU assembly kernels in CUDOLFINx, and then present a use case of complete GPU offloading for both assembly and linear solves. Examples, installation instructions and the source code of CUDOLFINx can be found at \href{https://github.com/bpachev/cuda-dolfinx}{github.com/bpachev/cuda-dolfinx}. The Navier--Stokes example is in a separate repository (\href{https://github.com/bpachev/dolfinx\_ipcs}{github.com/bpachev/dolfinx\_ipcs}). All computations in the following experiments are performed in double-precision.

\subsection*{Poisson}

%The classical introductory PDE is the Poisson equation, given by:
%\begin{align}
%  - \nabla^{2} u &= f \quad {\rm in} \ \Omega, \\
%  u &= 0 \quad {\rm on} \ \Gamma_{D}, \\
%  \nabla u \cdot n &= g \quad {\rm on} \ \Gamma_{N}.
%\end{align}
%Here $\Omega$ is the solution domain, $\Gamma_{D}$ is the Dirichlet boundary, and $\Gamma_{N}$ the Neumann boundary, with $f$ and $g$ some functions. Integration by parts leads to the variational problem:
%\begin{equation}
%\int_{\Omega}\nabla u \cdot \nabla v dx = \int_{\Omega}fv dx + \int_{\Gamma_n} gv ds,
%\end{equation}
%with $v$ a test function assumed to be zero on $\Gamma_D$. Solving the variational problem requires the assembly of a stiffness matrix on the left side, as well as a load vector on the right side. The dominant step in assembly is the stiffness matrix.

Here we consider the problem of assembling a stiffness matrix for the solution of the Poisson equation on the unit cube. We use linear Lagrange finite elements on four different uniform, tetrahedral meshes ranging in size from about 1.3 to 16.5 million elements.  For each mesh, Table~\ref{tab:poisson_results} reports the throughput in millions of degrees of freedom per second (MDOF/s) for assembling the stiffness matrix with CUDOLFINx on an NVIDIA H100 GPU and with DOLFINx on a 72-core NVIDIA Grace CPU. Examining the results, the GPU doesn't appear fully saturated for the first two meshes, both of which have significantly fewer than one million degrees of freedom. This suggests that problems with approximately one million degrees of freedom or greater are needed to maximize the benefits of GPU acceleration. In the original exploration of GPU acceleration for FEniCSx, on which this work is based, a throughput of 189\,MDOF/s was reported for the Poisson assembly problem using the Uniform 3 mesh on an NVIDIA V100 GPU \citep{trotter2023targeting}. In this study, we attained a throughput of 376\,MDOF/s for the same mesh with an H100 GPU. The improved performance is unsurprising given that the H100 is a newer model with more parallel cores and a higher memory throughput.
% \begin{table}[t]
%     \centering
% \begin{tabular}{llrr}
% \toprule
% Mesh & Solver & Cells & Vertices\\
% \midrule
% Uniform 1 & Poisson &  1,296,000 &   226,981 \\
% Uniform 2 & Poisson &  3,072,000 &   531,441 \\
% Uniform 3 & Poisson &  6,000,000 & 1,030,301 \\
% Uniform 4 & Poisson & 16,464,000 & 2,803,221 \\
% Tidal 1 & Shallow Water & 980,000 & 491,401 \\
% Tidal 2 & Shallow Water & 2,000,000 & 1,002,001 \\
% Tidal 3 & Shallow Water & 3,380,000 & 1,692,601 \\
% channel3D-lc003 & Navier--Stokes & 1,995,628 & 355,319 \\
% \bottomrule
% \end{tabular}
% \caption{List of meshes.}
%     \label{tab:meshes}
% \end{table}

\begin{table}[t]
    \centering
\begin{tabular}{lrrrr}
\toprule
          &          &             & \multicolumn{2}{l}{Matrix Assembly} \\
                                     \cmidrule(lr){4-5}
Mesh      & Elements & DOFs        & H100 (GPU) & Grace (CPU) \\
\midrule
Uniform 1 &  1,296,000 &   226,981 & 182.8 & 0.48 \\
Uniform 2 &  3,072,000 &   531,441 & 297.5 & 1.16 \\
Uniform 3 &  6,000,000 & 1,030,301 & 376.5 & 9.13 \\
Uniform 4 & 16,464,000 & 2,803,221 & 373.3 & 8.94 \\
\bottomrule
\end{tabular}
\caption{Performance of Poisson matrix assembly, in millions of DOFs per second.}
    \label{tab:poisson_results}
\end{table}

% \begin{table}[t]
%     \centering
% \begin{tabular}{lrr}
% \toprule
% Mesh & H100 GPU & Grace CPU \\
% \midrule
% Uniform 1 & 182.8 & 0.48 \\
% Uniform 2 & 297.5 & 1.16 \\
% Uniform 3 & 376.5 & 9.13 \\
% Uniform 4 & 373.3 & 8.94 \\
% \bottomrule
% \end{tabular}
% \caption{Performance of Poisson matrix assembly, in millions of DOFs per second.}
%     \label{tab:poisson_results}
% \end{table}

\subsection*{Shallow Water}

While the Poisson equation is a useful testscase, it does not represent the full range of complexity enabled by FEniCSx. For a more realistic example, we present some assembly benchmarks using the variational forms in SWEMniCS \citep{dawson2024swemnics}, a FEniCSx-based solver for the shallow water equations. SWEMniCS implements a suite of stablized two-dimensional shallow water solvers with implicit time stepping. Due to the nonlinearity of the shallow water equations, a Newton solve is required at each time step. We investigated the efficiency of automatically generated CUDA assembly kernels for two stabilized schemes within SWEMniCS: the Discontinuous Galerkin (DG), and the Streamline Upwind Petrov-Galerkin (SUPG) schemes. The DG method uses broken test and trial spaces with a Lax-Freidrichs numerical flux. The DG scheme is more numerically stable, but requires more degrees of freedom than SUPG due to the use of discontinuous function spaces. For each scheme, we averaged the performance of both GPU and CPU assembly kernels over twenty Newton iterations for tidal flow simulations on a set of square uniform triangular meshes. Results are presented for both assembly of the Jacobian matrix and the residual vector. In this experiment, an NVIDIA A100 GPU was compared to a 64-core AMD Epyc CPU.
Table \ref{tab:swe_a100_vs_epyc} summarizes the results of the experiment.
\begin{table}[t]
    \centering
    \begin{tabular}{lrrrrrr}
\toprule
        &           &           & \multicolumn{2}{c}{DG} & \multicolumn{2}{c}{SUPG} \\
                                  \cmidrule(lr){4-5}       \cmidrule(lr){6-7}
Mesh    &  Elements &  Vertices & Jacobian & Residual    & Jacobian & Residual \\
\midrule
Tidal 1 &   980,000 &   491,401 &     0.98 &        9.21 &     5.36 &     6.78 \\
Tidal 2 & 2,000,000 & 1,002,001 &     0.99 &       12.18 &     5.51 &     6.84 \\
Tidal 3 & 3,380,000 & 1,692,601 &     0.95 &        9.66 &     5.36 &     5.60 \\
\bottomrule
\end{tabular}
    \caption{Speedup of assembly kernels on an NVIDIA A100 GPU relative to a 64-core AMD Epyc CPU. Larger numbers indicate faster GPU runtimes.}
    \label{tab:swe_a100_vs_epyc}
\end{table}


% \begin{table}[t]
%     \centering
%     \begin{tabular}{lrrrr}
% \toprule
%         & \multicolumn{2}{c}{DG} & \multicolumn{2}{c}{SUPG} \\
%           \cmidrule(lr){2-3}       \cmidrule(lr){4-5}
% Mesh    & Jacobian & Residual    & Jacobian & Residual \\
% \midrule
% Tidal 1 &     0.98 &        9.21 &     5.36 &     6.78 \\
% Tidal 2 &     0.99 &       12.18 &     5.51 &     6.84 \\
% Tidal 3 &     0.95 &        9.66 &     5.36 &     5.60 \\
% \bottomrule
% \end{tabular}
%     \caption{Speedup of assembly kernels on an NVIDIA A100 GPU relative to a 64-core AMD Epyc CPU. Larger numbers indicate faster GPU runtimes.}
%     \label{tab:swe_a100_vs_epyc}
% \end{table}

Interestingly, the speedups differ for each variational form. The DG residual vector assembly has a much better speedup factor than the SUPG kernels, while the Jacobian matrix assembly is much worse. On the other hand, SUPG has consistent speedups for both residual and Jacobian matrix assembly.
Using NVIDIA's Nsight Compute profiler, we can better understand the difference in performance between DG and SUPG.
Profiler results for the ``Tidal 3'' testcase are presented in Table \ref{tab:tidal_prof}. The profiler reports \textit{occupancy}, which indicates the percentage of active threads on the GPU relative to the total GPU thread capacity. Occupancy can be limited by the resources needed per thread, such as shared memory or registers. It can also be impacted by excessive branching or other kernel design flaws. In our case, the generated assembly kernels required a high number of registers, which limited the occupancy to under 20\%. However, both the DG and SUPG kernels were able to utilize a large fraction of the device throughput - memory in the case of DG and compute in the case of SUPG.
\begin{table}[t]
    \centering
\begin{tabular}{llrrrr}
\toprule
Method & Kernel & \begin{tabular}{@{}l}Theoretical\\ Occupancy\end{tabular} & \begin{tabular}{@{}l}Achieved\\ Occupancy\end{tabular} & \begin{tabular}{@{}l}Memory\\ Throughput\end{tabular} & \begin{tabular}{@{}l}Compute\\ Throughput\end{tabular} \\
\midrule
DG   & Jacobian & 12.50\,\% & 12.33\,\% & 41.44\,\% & 10.43\,\% \\
DG   & residual & 18.75\,\% & 18.74\,\% & 64.97\,\% & 18.20\,\% \\
SUPG & Jacobian & 12.50\,\% & 12.33\,\% & 48.66\,\% & 59.31\,\% \\
SUPG & residual & 12.50\,\% & 12.39\,\% &  5.64\,\% & 77.85\,\% \\
\bottomrule
\end{tabular}
\caption{Profiling statistics for the GPU assembly kernels on a square mesh.}
    \label{tab:tidal_prof}
\end{table}

\subsection*{Navier--Stokes}

While matrix assembly is a crucial part of the finite element method, most solvers also require the solution of linear systems. To assess the practicality of offloading a typical FEniCSx code to the GPU, we enhanced an existing FEniCSx Navier--Stokes solver \citep{dokkenipcs} with our extension. The original solver solves the Navier--Stokes equations using an incremental pressure correction scheme \citep{dokken2019shape}, which requires three stages per time step. While each stage requires a linear solve and assembly of a vector, only the first requires reassembly of a stiffness matrix. Second-order Lagrange tetrahedral elements were used for the velocity field, and first-order Lagrange elements for the pressure field. The mesh was a refined version of the 3D channel with an obstacle used in the original code, and contained 1,995,628 tetrahedra with 355,319 vertices.

We compared the runtimes of the original solver on a 72-core Grace CPU to the CUDA-accelerated version using an H100 GPU. Average timing results over 100 time steps for each component of the solver are provided in Table \ref{tab:navier_stokes_results}.
\begin{table}[t]
    \centering
\begin{tabular}{llrrrr}
\toprule
Stage & Solver/PC & GPU Assembly & GPU Solve & CPU Assembly & CPU Solve \\
\midrule
1 & BCGS/Jacobi     & 0.348 & 1.601 & 0.906 & 6.339 \\
2 & GMRES/BoomerAMG & 0.009 & 0.013 & 0.007 & 0.108 \\
3 & CG/Jacobi       & 0.004 & 0.079 & 0.013 & 0.564 \\
\bottomrule
\end{tabular}
\caption{Runtime for each stage of the Navier--Stokes solver in seconds.}
    \label{tab:navier_stokes_results}
\end{table}
Overall, the CUDA-accelerated solver averaged 2.07\,s per time step, while the original solver took 7.89\,s per time step. While the solution time was dominated by the linear solve for the first stage, assembly comprised over 10\% of the total runtime for both the GPU and CPU codes. We note that compared to the Poisson example, the assembly speedup for the Navier--Stokes code is smaller. We hypothesize this is due to the use of second-order elements for the velocity field. Higher order elements are known to pose difficulties for the GPU offloading approach used within CUDOLFINx because the generated tabulate tensor kernels can require a large number of registers. This results in a phenomenon known as \textit{register spilling}, in which the GPU kernels are forced to utilize global memory to store some local variables, which degrades performance. Solutions include retooling of the generated code to require less constant memory, or assembling each element with multiple GPU threads to increase the available number of registers. Both of these solutions would require substantial enhancements to FFCX, the form compiler within the FEniCS project. Nevertheless, the attained speedup for assembly with quadratic elements in this case is still significant.

\section*{Conclusion}

We have introduced CUDOLFINx - an add-on enhancement to FEniCSx which provides GPU-acclerated assembly routines. We've demonstrated that offloading assembly to a GPU can provide significant speedups compared to multi-core CPUs, particularly for first-order elements. The package has a simple Python interface, and can be utilized in FEniCSx codebases with little effort.

In the near future, we intend to add support for multiple GPUs. Subsequent efforts will focus on providing support for non-NVIDIA GPUs, as well as developing efficient kernels for assembly with high-order elements. Work is currently underway to expand the documentation and examples, as well as to create easily installable binary distributions.

%Sample references \citep{alnaes2015fenics, baratta2023dolfinx}.
%Source code for this book is found at \href{https://github.com/meg-simula/2024-fenics-proceedings}{github.com/meg-simula/2024-fenics-proceedings}


\begin{acknowledgement}
  We gratefully acknolwedge the use of the Lonestar6 and Vista systems at the Texas Advanced Computing Center under the "ADCIRC" allocation.
\end{acknowledgement}

\bibliographystyle{spbasic}
% Write the full path of your bibfile relative to book.tex
\bibliography{chapters/chp1/bibliography.bib}
